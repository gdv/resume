% !TEX program = latexmk -xelatex
\documentclass[11pt,a4paper,roman]{moderncv}
\moderncvstyle{banking}
\moderncvcolor{blue}
%\renewcommand{\familydefault}{\rmdefault}
%\nopagenumbers{}
\usepackage{verbatim}


% adjust the page margins
\usepackage[scale=0.75]{geometry}
%\setlength{\hintscolumnwidth}{3cm}                % if you want to change the width of the column with the dates
%\setlength{\makecvtitlenamewidth}{10cm}           % for the 'classic' style, if you want to force the width allocated to your name and avoid line breaks. be careful though, the length is normally calculated to avoid any overlap with your personal info; use this at your own typographical risks...


% font loading
% for luatex and xetex, do not use inputenc and fontenc
% see https://tex.stackexchange.com/a/496643
\ifxetexorluatex%
  \usepackage{fontspec}
  \usepackage{unicode-math}
  \defaultfontfeatures{ Scale=MatchLowercase, Ligatures=TeX }
  \setmainfont{TeX Gyre Termes}[Scale=1.0]
  \setmathfont{TeX Gyre Termes Math}
\else
  \usepackage[utf8]{inputenc}
  \usepackage[T1]{fontenc}
  \usepackage{newpxtext}
  \usepackage[vvarbb]{newpxmath}
%  \usepackage{lmodern}
\fi

% personal data
\name{Gianluca}{Della Vedova}
\email{gianluca.dellavedova@unimib.it}
\homepage{https://www.unimib.it/gianluca-della-vedova}
\social[github]{gdv}
\social[orcid]{0000-0001-5584-3089}
%\social[googlescholar]{0gaIFokAAAAJ}


% to show numerical labels in the bibliography (default is to show no labels); only useful if you make citations in your resume
\makeatletter
\renewcommand*{\bibliographyitemlabel}{\@biblabel{\arabic{enumiv}}}
\makeatother
\usepackage[resetlabels,labeled]{multibib}
\newcites{I}{Chapters in Book}
\newcites{J}{Journal articles}
\newcites{C}{Conference proceedings}
\newcites{P}{Preprints}

\title{CV}
\begin{document}
\makecvtitle%

\section{Current position}

\noindent
  Associate professor with tenure (SSD INF/01 --- Informatica)\\
  Dipartimento di Informatica, Sistemistica e Comunicazione\\
  Università degli Studi di Milano -- Bicocca

\section{Positions}

  \cventry{10/2012--today}{Dept. of Informatics, Systems, and Communications}{Associate Professor}{Univ. Milano -- Bicocca}{}{}
  \cventry{10/2005--09/2012}{School of Statistics}{Associate Professor}{Univ. Milano -- Bicocca}{}{}
  \cventry{05/2001--09/2005}{School of Statistics}{Assistant Professor}{Univ.Milano -- Bicocca}{}{}

\section{Habilitation}

  \cventry{08/2018--today}{Italian habilitation to be Full professor of Computer Science}{Abilitazione a Professore di I fascia, settore concorsuale 01/B1}{}{}{}

\section{Education}

  \cventry{1996--2001}{}{Ph.D. in Computer Science}{Università degli Studi di
  Milano}{}{Thesis: ``Multiple Sequence Alignment and Phylogenetic Reconstruction:
  Theory and Methods in Biological Data Analysis''}{}

  \cventry{1990--1995}{}{M.Sc. in Computer Science}{Università degli Studi di Milano}{}{Thesis: ``Sequential and Parallel Algorithms for Graph Decomposition''}{}

\section{Funding}
\subsection{Leading}


  \cvitem{2020--2023}{%
  H2020-MSCA-RISE-2019 (Univ. Milano -- Bicocca amount: 197 800€)
  Pan-genome Graph Algorithms and Data Integration (PANGAIA) ---
  Horizon 2020 Marie Skłodowska-Curie Research and Innovation Staff Exchange
  programme.}

This project is coordinated by a co-leader of my research lab and it involves a
post-doc I supervise (Simone Ciccolella), another post-doc a co-supervise, and 2
PhD students (that I co-supervise).

The main goal of this project is to foster collaborations between different
research groups in computational pangenomics.
I have been one of the main participants in planning and writing the proposal. I am
\textbf{leading WP5} Communication and Dissemination.

The project involves Universität Bielefeld (Germany), Centrum Wiskunde \& Informatica (the Netherlands), The
Institut Pasteur (France), Comenius University Bratislava (Slovakia), Geneton s.r.o.
(Slovakia), Cornell University (USA),  University of Tokyo (Japan),
Simon Fraser University (Canada), Pennsylvania State University (USA),
University of California (USA).
I have developed an especially strong collaboration with Prof. Iman Hajirasouliha at Cornell
University on tumor phylogeny inference and metagenomics where also Simone
Ciccolella is actively working.

  \cvitem{2021--2024}{%
  H2020-MSCA-ITN (Univ. Milano -- Bicocca amount: 261 499,68€) ALgorithms for PAngenome
  Computational Analysis (ALPACA) ---
  Horizon 2020 Marie Skłodowska-Curie Innovative Training Network.
%and I am a \textbf{member of the Executive Committee}.
}

As part of the activities of this project I am \textbf{co-supervising} a PhD student.
The project involves Universität Bielefeld (Germany), Centrum Wiskunde \& Informatica (the Netherlands), The
Institut Pasteur (France), Comenius University Bratislava (Slovakia), Geneton s.r.o.
(Slovakia), University of Helsinki (Finland), European Molecular Biology
Laboratory --- European Bioinformatics Institute,
Heinrich-Heine-Universität Düsseldorf (Germany),
Università di Pisa (Italy),
CNRS (France),
University of Cambridge (UK),
INRIA (France),
Pendulum Therapeutics, Inc. (USA),
Vrije Universiteit Amsterdam (the Netherlands),
Oxford Nanopore Technologies plc (UK),
DNANexus (US),
Sorbonne University (France),
Deinove (France),
Finnish Red Cross (Finland),
BaseClear B.V. (the Netherlands),
French Alternative Energies and Atomic Energy Commission (France),
Cornell University (USA),  University of Tokyo (Japan),
Simon Fraser University (Canada), Pennsylvania State University (USA),
University of California (USA).


  \cvitem{2013--2016}{%
  Modulation of anti-cancer immune response by
  regulatory non-coding RNAs --- Fondazione Cariplo 2013.
  I have been in \textbf{charge of the bioinformatics WP}. The project had two
  research units: the National Institute of Molecular Genetics and the Univ.
  Milano -- Bicocca.
  }

During this project, I have been the \textbf{scientific supervisor of two
  1-year post-docs} and I have co-supervised a third postdoc.

  \cvitem{2011--2014}{%
  MIUR/Regione Lombardia 2011 (Univ. Milano -- Bicocca amount: 199 991€).
  Piattaforma di Analisi TRaslazionale Integrata (PATRI).
  I have been \textbf{in charge of all bioinformatics aspects}.
  }

This project has led to a 1-year postdoc (23 000€) and 3 short-term research
  contracts (each between 10 000€ and 12 000€).
  I have co-supervised the postdoc and I have been the only supervisors of all
  short-term contracts.


  \cvitem{2016}{%
  Fondo di Ateneo 2016 (12 490€). Modelli computazionali e algoritmi:
  aspetti teorici e sperimentali, con applicazioni alla Bioinformatica.
  \textbf{PI}.
  }
  \cvitem{2015}{%
  Fondo di Ateneo 2015 (10 980€). Algoritmi combinatori e modelli di
  calcolo: aspetti teorici e applicazioni in Bioinformatica.
  \textbf{PI}.
  }
  \cvitem{2014}{%
  Fondo di Ateneo 2014 (12 186€). Algoritmi e modelli computazionali:
  aspetti teorici e applicazioni nelle scienze della vita.
  \textbf{PI}.
  }
  \cvitem{2013}{%
  Fondo di Ateneo 2013 (9 337€). Metodi algoritmici e modelli: aspetti
  teorici e applicazioni in bioinformatica. \textbf{PI}.
  }
  \cvitem{2011}{%
  Fondo di Ateneo 2011 (4 055€). Tecniche algoritmiche avanzate in
  Biologia Computazionale. \textbf{PI}.
  }
  \cvitem{2006}{%
  Grandi Attrezzature 2006 (40 000€). Laboratorio Virtuale
  Statistico-Territoriale --- Virtual Statistical--Territorial Laboratory.
  This was a joint project between the Statistics Department and the Dept. of
  Sociology and I have been one of the two people \textbf{in charge of the entire
  project} (and the only one from the Statistics Department). It has resulted in the acquisition of 2 servers that are used to
  provide services to research and teaching activities on territorial sociology
  and computational statistics.
  \textbf{co-PI}.
  }
  \cvitem{2005--2008}{%
  MIUR/PRIN 2005, Potenzialità e ottimizzazione delle banche dati
  automatizzate in epidemiologia.
  \textbf{In charge of all bioinformatics aspects}.
  }

\subsection{As a Participant}


\cvitem{2013--2016}{%
  Regione Lombardia. SPAC3 --- Servizi smart della nuova Pubblica
  amministrazione per la Citizen-Centricity in cloud.}
  \cvitem{2011--2014}{%
  MIUR/PRIN 2011
  Automi e Linguaggi Formali: Aspetti Matematici e Applicativi}
  \cvitem{2003}{%
  MIUR/FIRB 2003. Bioinformatica per la Genomica e la Proteomica.}
  \cvitem{2000--2001}{%
  NSF CCR-9988353, ITR-0085910.}
  \cvitem{1999--2001}{%
  MURST COFIN 98 ``Bioinformatica e ricerca genomica''.}
  \cvitem{1994--1995}{%
  MURST 40\% ``Algoritmi e strutture di calcolo''.}
  \cvitem{1994--1995}{%
  ESPRIT-BRA ASMICS 2 n.~6317.
  }



\section{Teaching}

I have started teaching when I have been hired as an Assistant professor at
  the School of Statistics in 2001.
  I was the first Computer Science faculty of the school, therefore I had to
  design new courses tailored to students with a good mathematical background,
  but with no Computer Science experience.
  During my career I had to design a course several times, partly due to the
  evolution of the background of the students and to changes in the overall
  degree in Statistics and Statistical Economics, partly because I had to teach
  also to students majoring in Computer Science or in Data Science.

  My teaching philosophy is based on active learning. This requires important
  and continuous interactions between teacher and students, requiring the
  development of problems that students have to attack, as a group or solo.
  Designing an active learning course is usually innovative, since existing
  courses are mostly designed for more traditional approaches, where students
  have a more passive role and acquiring problem solving skills is not a main focus.

  \subsection{Ph.D. courses taught}


\cvitem{2016, 2018, 2020}{\emph{Advanced Algorithms}, PhD in Computer Science, Univ.
Milano -- Bicocca.
  I have designed the course, since it had never been taught before.
}

\cvitem{2021, 2023}{\emph{Graph Theory and Algorithms}, PhD in Computer Science, Univ.
Milano -- Bicocca.
  I have co-designed the course, since it had never been taught before.
}

\subsection{Courses taught --- in charge of the course}

\cvitem{2022--today}{%
    \emph{Large-Scale Graph Algorithms},
    M.Sc. in Computer Science,
Univ. Milano -- Bicocca (6 ECTS).
  I have designed the course, since it had never been taught before.
}

\cvitem{2021--today}{%
    \emph{Laboratorio di Informatica (Computer Science Lab)},
    B.Sc. in Statistics, B.Sc. in Statistical Economics,
Univ. Milano -- Bicocca (3 ECTS).
I had fully redesigned the course.
}


\cvitem{2017-today}{%
\emph{Foundations of Computer Science}, M.Sc. in Data Science,
Univ. Milano -- Bicocca. (6 ECTS).
  I have designed the course, since it had never been taught before.
}
\cvitem{2014-today}{%
    \emph{Elementi di Bioinformatica (Elements of Bioinformatics)}, B.Sc. in Computer Science, Univ.
Milano -- Bicocca (8 ECTS).
  I have designed the course, since it had never been taught before.
  }
\cvitem{2001--2020}{%
    \emph{Basi di Dati (Databases)}, B.Sc. in Statistics, B.Sc. in Statistical Economics,
Univ. Milano -- Bicocca (6 ECTS).}

\cvitem{2007-2022}{%
    \emph{Bioinformatica (Bioinformatics)}, M.Sc. in Biostatistics, Univ.
Milano -- Bicocca (6 ECTS).
  I have designed the course, since it had never been taught before.
  }

\cvitem{2010--2013}{%
    \emph{Algoritmi su stringhe (Text Algorithms)}, B.Sc. in Computer Science, Univ.
Milano -- Bicocca.
  I have designed the course, since it had never been taught before.
  }
\cvitem{2008}{%
    \emph{Strumenti informatici per la statistica M (Computational Tools for
    Statistics)}, M.Sc. in
Biostatistics, blended e-learning, Univ. Milano -- Bicocca.
  I have designed the course, since it had never been taught before.
  }
\cvitem{2007--2009}{%
    \emph{Informatica Applicata S (Applied Computer Science)}, M.Sc. in Biostatistics (2
ECTS), Univ. Milano -- Bicocca.
  I have designed the course, since it had never been taught before.
  }
\cvitem{2001--2008}{%
    \emph{Laboratorio Statistico-Informatico (Computational Statistics Lab)}, all B.Sc. programs of the School
of Statistics (6 ECTS), Univ. Milano -- Bicocca.
  I have designed the course, since it had never been taught before.
  }
\cvitem{2001--2008}{%
    \emph{Programmazione e Basi Dati (Programming and Databases)},
all B.Sc. programs of the School of Statistics
(6 ECTS), Univ. Milano -- Bicocca.
  I have designed the course, since it had never been taught before.
  The course has changed name to "Basi di Dati".
  It was the first database course in Italy designed for students in
  Statistics.
  }
\cvitem{2008}{%
    \emph{Fondamenti di Informatica (Fundamentals of Computer Science)}, M.Sc. in Biostatistics, Univ.
Milano -- Bicocca.
  I have designed the course, since it had never been taught before.
  }
\cvitem{2006}{%
    \emph{Laboratorio di Informatica (Computer Science Lab)},
all B.Sc. programs of the School of Statistics
(6 ECTS), Univ. Milano -- Bicocca.
  I have designed the course, since it had never been taught before.
  }

\subsection{Postgraduate courses}

\cvitem{2021}{%
    \emph{Beyond Genome Assembly},
  Master   ``qOmics: quantitative methods for Omics Data'',
Univ.
Milano -- Bicocca and Univ. Pavia.}

\cvitem{2012, 2014}{%
    \emph{Data Base e Sistemi Informativi (Databases and Information
    Systems)},
Master di primo livello in
Amministratore di Sistema per la Diagnostica per Immagini (Master in
Image Diagnostics Systems Administrator),
Univ.
Milano -- Bicocca.}

\cvitem{2007, 2010, 2011}{%
    \emph{Fondamenti di Informatica (Fundamentals of Computer Science)},
Master di primo livello in
Amministratore di Sistema per la Diagnostica per Immagini (Master in
Image Diagnostics Systems Administrator),
Univ.
Milano -- Bicocca.}
\cvitem{2003}{%
\emph{Fondamenti di Informatica e Elementi di Programmazione (Fundamentals
of Computer Science and Elements of Programming)}, Master di primo
livello in Bioinformatica
(Master in Bioinformatics), Univ. Milano -- Bicocca.}
\cvitem{2001--2002}{%
    \emph{Sistemi Informatici e Elementi di Programmazione (Introduction to
    Programming)}, Master di primo
livello in Bioinformatica (Master in Bioinformatics), Univ. Milano -- Bicocca.}

\section{Supervision}

\subsection{Postdocs}

I have been \textbf{scientific supervisor} of the following postdoctoral
positions:

\cvitem{2022--today}{%
Simone Ciccolella. 2-year postdoc position on \emph{Computational Approaches for
tumor and viral phylogenies}.
}

\cvitem{2018--2019}{%
Murray Patterson. 2-year postdoc position on \emph{Haplotype
assembly from sequencing reads}. Currently Assistant Professor
(Tenure-track) at Georgia State University.
}

\cvitem{2015}{%
Hassan Mahmoud Mohamed Ramadan Mohamed.
1-year postdoc position on
\emph{Methodology for treatment and data analysis of NGS data for
the detection of alternate splicing events}.
}

Moreover, I have co-supervised the following postdocs:

\cvlistitem{Luca Denti}
\cvlistitem{Marco Previtali}
\cvlistitem{Stefano Beretta}

\subsection{PhD students supervision}

\textbf{Supervisor} of


\cvitem{2021--2024 (expected)}{%
  Jorge Avila Cartes --- PhD in Computer Science}

\cvitem{2017--2021}{%
  Simone Ciccolella, ``\emph{Algorithms for cancer phylogeny inference}'' --- PhD in
  Computer Science.
  Currently Post-doc at Università di Milano -- Bicocca.}

\cvitem{2013--2017}{%
  Marco Previtali, ``\emph{Self-indexing for de novo assembly}'' --- PhD in Computer Science.
Currently at Bloomberg}

\cvitem{2008--2012}{%
  Stefano Beretta, ``\emph{Algorithms for Next Generation Sequencing Data
  Analysis}'' --- PhD in Computer Science.
  Currently at San Raffaele Telethon Institute for Gene Therapy}

\cvitem{2001--2004}{%
  Riccardo Dondi, ``\emph{Computational Problems in the Study of Genomic
  Variations}'' --- PhD in Computer Science.
  Currently Associate Professor at  Università di
  Bergamo}

Moreover, I have actively contributed to the training of the following
PhD students:

\cvlistitem{%
    Giulia Bernardini --- PhD in Computer Science.
    Currently Assistant Professor at Univ. Trieste.}
\cvlistitem{%
  Simone Zaccaria --- PhD in Computer Science.
  Currently   Group Leader presso Department of   Oncology, Univ. College London}
  \cvlistitem{%
  Anna Paola Carrieri --- PhD in Computer Science.
  Currently   Research Staff Member presso IBM
  Research Lab, UK}

\subsection{M.Sc. students supervision}

  I have supervised or co-supervised the thesis of more than 10 M.Sc.
  students  in Computer Science, Statistical Economics, Biostatistics,
  Data Science at University of Milano--Bicocca and at the University of Milan.
  I am currently supervising the thesis of a M.Sc. in Bioinformatics at the
Pwani University Biosciences Research Centre, Kilifi (Kenya), as the result of a
long-standing collaboration with Pjotr Prins (University of Tennessee).




  I have supervised the activities related to the
  \textbf{Exchange mobility EXTRA UE} program of:

  \cvlistitem{%
  Ramesh Rajaby ha spent 6 months with the research group led by
    prof. Jesper Jansson (Kyoto University, Japan). Ramesh Rajaby is
    currently postdoc at  National University of Singapore.}
    \cvlistitem{%
    Simone Ciccolella has spent 3 months with the research group led by
prof. Iman Hajirasouliha (Weill Cornell Medicine, New York).
Simone Ciccolella is currently a postdoc under my supervision.}

  \subsection{B.Sc. students supervision}

  I have supervised the final projects of more than 50 B.Sc. students in
  Computer Science, Statistics, and Statistical Economics.

\section{Awards}

\subsection{Best Poster Awards}

\cvitem{Recomb 2018}{Luca Denti, Raffaella Rizzi, Stefano Beretta, Gianluca Della Vedova, Marco Previtali, and Paola Bonizzoni.
\emph{ASGAL: Aligning RNA-Seq Data to a Splicing Graph to Detect Novel Alternative Splicing Events}}

\section{Professional activities}

\subsection{Steering Committee Member}


\cvitem{2018-today}{%
  \textbf{Steering Committee member} of Computability in Europe (CiE).
  During the same period I have also been the Executive Officer of the Steering Committee.
}



\subsection{Program Committee Chair}

\cvitem{2023}{%
  \textbf{Program Committee Chair} of Computability in Europe (CiE) 2023.
This international conference will take place in Batumi (Georgia) in July 24-28
and is the flagship conference of the Association Computability in Europe that
has more than 1000 members and focuses on computability science in all its
multidisciplinary facets.
The conference proceedings will be published in the LNCS series.
}

  \cvitem{2021}{%
  \textbf{Chair} of the Data Structure in Bioinformatics Workshop (DSB 2021),
  February 11-12, 2021.
The workshop had 19 research talks given from international scholars.
}

\cvitem{2021}{%
  \textbf{Organizer}, of Pangenome Bio Hacking (PGBH) 2021 (co-organized with
  Erik Garrison, Enza Colonna, Pjotr Prins).
This online conference has taken place on December 9-10, 2021, and we have
invited 9 speakers from Europe and the US.
It has been the first world-wide conference for researchers who have a focus on pangenomics free and open source software development.
  }

  \cvitem{2020}{%
  \textbf{Organizer}, of the Computability in Europe special session ``\emph{Large Scale
  Bioinformatics and Computational Sciences}'' (co-organized with Iman
  Hajirasouliha, Weill-Cornell Medicine, USA).
We have selected and invited four speakers who have given an invited talk.
}


\subsection{PhD schools Board}


\cvitem{2022}{%
  \textbf{Supervisory Board Member}, PhD Summer School ``Introduction to
  Pangenomics'',
  held at the Lake Como School of Advanced Studies, July 4-8, 2022.
The school has been attended by 25 international PhD students and had 12
teachers from Europe and the US.
The Supervisory Board had 3 members who have decided the full program of the school.
  }

\subsection{Program Committee Member}

\cvitem{Program Committee Member}{%
  \begin{itemize}
  \item Computability in Europe (CiE) 2013, 2019, 2020, 2022
  \item Workshop on Algorithms in Bioinformatics (WABI) 2020, 2022
  \item ISCB European Conference on Computational Biology (ECCB 2019)
  \item Combinatorial Pattern Matching (CPM 2019)
  \item Symposium on String Processing and Information Retrieval (SPIRE 2017)
  \item Bioinformatics Open Source Conference (BOSC) from 2016 to 2021.
        \end{itemize}
  }


\cvitem{2020}{%
  \textbf{Editor} of the proceedings of the conference Computability in Europe 2020.
The proceedings have been published in the LNCS 12098 volume, Springer-Verlag.
  }

\subsection{Editorial Board Member}

\cvitem{2016--2019}{%
  \textbf{Editorial Board} member of the journal
  ``Advances in Bioinformatics''.
}

\cvitem{2004}{%
  \textbf{special issue editor}, Journal of Computer Science and
  Technology.}


\subsection{PhD defence committee member}
\cvitem{2023}{%
  Evaluation committee member, PhD thesis in Computer Science, Univ. Milano.
  }

\cvitem{2018}{%
  Evaluation committee member, PhD thesis in Computer Science, Univ. Roma la Sapienza.
}

\cvitem{2018}{%
  Evaluation committee member, PhD thesis in Computer Science, Univ. Milano.
  }


\subsection{Competition evaluation member}


  \cvitem{2017}{%
  Evaluation committee member, competition for an Assistant Professor position
  (valutazione comparativa per RTDb), Univ. Milano
  }

  \cvitem{2006}{%
  Evaluation committee member, competition for an Assistant Professor position
  (valutazione comparativa per ricercatore universitario), Univ. Milano
  }

\subsection{Incoming visits}

\cvitem{2018}{%
I have been in charge of hosting Prof. Vladimir Filipovic (Associate Professor at University of Belgrade, Faculty of Mathematics,
Department of Computer Science) during his sabbatical from 20/01/2018 to 30/09/2018.
 During his visit, we have collaborated on the development of metaheuristics in Bioinformatics
}

\subsection{Reviewer Service}

I have reviewed papers for the following journals:
\begin{itemize}
\item ACM/IEEE Transactions on Computational Biology and Bioinformatics
\item Algorithmica
\item Algorithms
\item Bioinformatics
\item Briefings in Bioinformatics
\item Graphs and Combinatorics
\item Information Processing Letters
\item Journal of Computational Biology
\item INFORMS J. Computing
\item Journal of Computer Science and Technology
\item Theoretical Computer Science
\item Theory of Computing Systems
\end{itemize}


  \subsection{University Service}

  \cvitem{2021-today}{%
  \textbf{Scientific Committee Member}
  Postgraduate Master Programme ``quantitative methods for Omics Data'' (EQF
  level 8),
  University of Milano -- Bicocca
  }

\cvitem{2020-today}{%
  \textbf{Representative of the Università degli Studi di Milano -- Bicocca}
  Joint Research Unit ELIXIR IIB General Assembly,
  Italian Node of Elixir Europe, the main intergovernative Europan
  organization in Bioinformatics.
}

\cvitem{2019-today}{%
  \textbf{Deputy coordinator of the PhD program} in Computer Science,
  Univ. Milano -- Bicocca
}

\cvitem{2018-today}{%
  \textbf{Quality Assurance}, M.Sc. in Data
  Science.
}

\cvitem{2019-today}{%
      \textbf{President of the Teaching Committee (Commissione
      Didattica)}, M.Sc.
  in Data Science.
}

\cvitem{2018-today}{%
  \textbf{PI} of Università degli Studi di Milano -- Bicocca
  of the MoU (Convenzione quadro) with Istituto Nazionale di Genetica
  Molecolare (National Institute of Molecular Genetics).
}

\cvitem{2020-today}{%
  \textbf{Scientific Committee Member}, Bicocca Ambiente Società
  Economia, by decree of the rector.
}

\cvitem{2018-today}{%
      Member of the \textbf{Students-Professors Committee (Commissione
      Paritetica Docenti-Studenti)},
  Dipartimento di Informatica, Sistemistica e Comunicazione
  }\cvitem{2013-today}{%
      \textbf{Board Member}, PhD program in Computer Science,
  Univ. Milano -- Bicocca
  }

  \cvitem{2011-today}{%
      \textbf{Executive Committee (Comitato direttivo)} Member,
      Centro di Produzione Multimediale di Ateneo (Multimedia Production
      Center),
      by decree of the Academic Senate.
    }

\cvitem{2015--2020}{%
  Local Technical Coordinator,
  Univ. Milano -- Bicocca,
  Elixir IIB
  }

\cvitem{2016-today}{%
  \textbf{In charge of Elixir IIB training activities} in Univ. Milano
  -- Bicocca.
  I have organized course on ``Genome Assembly and Annotation'', ``Data
  Carpentry Workshop'', ``Software Carpentry Workshop'', ``Docker
  Advanced'', and ``Exome analysis with Galaxy''.
  }
  \cvitem{2002--2012}{%
  \textbf{Representative of the Computer Science Area},
  School of Statistics.
  }

  \cvitem{2004--2012}{%
  \textbf{Representative of the School of Statistics},
  University Committee on Computer Science.
  }

  \cvitem{2010--2012}{%
  \textbf{Delegate of the Dean on e-learning}
  School of Statistics.
  }

  \cvitem{2007--2012}{%
  \textbf{In charge of the Computer Labs}
  School of Statistics.
  }

  \cvitem{2010--2012}{%
  Member of the e-learning committee,
  School of Statistics.
  }

  \cvitem{2002--2004}{%
  \textbf{Representative of the Computer Science Area},
  School of Statistics Committee in charge of designing the B.Sc.
  program in Statistics
  (Statistica e Gestione delle Informazioni).
  The program is still active.
  }\cvitem{2004}{%
      \textbf{Representative} of the School of Statistics for the short
      course
  ``Information Technology For Problem Solving (IT4PS)'', organized by
  the CRUI Foundation.
  }\cvitem{2003}{%
  \textbf{In charge} of the course 167388 ``Laboratorio Complementare di
  Informatica per Statistici (Elective Computer Science Lab for
  Statistics)'', part of the FSE 156165 project
  ``Progetto Quadro Università degli Studi di Milano -- Bicocca''.
  The course focused on intermediate Computer Science skills for
  statistics students.
  The course has been fully funded by the European Commission.
  }


  \section{Speaker}

  \cvitem{GGI Seminar Series, March 2022}{Invited by Department of Genetics, Genomics and Informatics, University of Tennessee. Title of the talk: ``Computational Aspects of Models of Evolution''}


  \cvitem{invited speaker}{Computability in Europe 2017 special session  ``Algorithmics for Biology''}

  \cvitem{speaker at ISMB 2001}{I have presented a paper as first author ---
  and the only author of my University --- to the conference Intelligent
  Systems for Molecular Biology 2001: the main Bioinformatics scientific conference (\textbf{CORE: A, Microsoft Academic: A+})}


\section{Research activity}


I have published more than \textbf{50 journal papers} with more than \textbf{1800
citations} and \textbf{h-index 20} (from Google Scholar).
According to Google Scholar, 7 of my papers have at least 100 citations each, and I have 166 coauthors.

From 2016 to 2019 I have been the founder and the leader of the Research Lab
``AlgoLab --- Experimental Algorithmics Lab'', at the Università di Milano -- Bicocca.
The research lab has several international collaborations and it focuses on
algorithm design, implementation, and experimental analysis on large datasets.
In fact, the research activities regards both methodological and experimental
aspects, leading to the development of software tools to attack bioinformatics
problems.
In 2019 the lab has merged with the Bioinformatics Lab, resulting in the
Bionformatics and Experimental Algorithmics (BIAS) Lab which I am effectively
co-leading with Paola Bonizzoni.


My personal research activity has focused on the development of combinatorial
algorithms in Bioinformatics, with a strong attention on foundational aspects.
Moreover, I have paid special attention to algorithm implementation and validation, from both a theoretical and an experimental point of view.


Bioinformatics algorithmics has gained relevance with the establishment of the
Human Genome project whose main goal is to determine the influence that
biological sequences (such as DNA, RNA) have on living beings.
Completing the sequencing of the human genome was only the first
fundamental step; finding which proteins are expressed by each gene and
determining the interaction among the various DNA sites are among the
most important open problems. The huge quantity of data to
analyze (remember that human DNA contains 3 billions nucleotides) makes
computers and efficient algorithms a needed cornerstone of the field.
My research has been mainly devoted to the design of such
efficient algorithms and can be detailed in a few subfields.

\subsection{Sequence comparison}\label{sequence-comparison}

The central dogma in Computational Biology states that sequence homology
leads to functional homologies (that is similarity among the effects
performed by such sequences), therefore it is of the utmost importance
to have some computational tools for comparing sequences. In this
direction the multiple sequence alignment problem has been formally
introduced in (Altschul, Lipman SIAM J. Appl. Math. 1989), even though
it had already been studied previously, and some of its variant are
intractable. My focus is on the study of the computational and approximation
complexity and on the development of efficient algorithms for the classical
notions of longest common subsequence and shortest common supersequence and for
sequence alignment.

Most recently, my focus is on computational pangenomics, where a large set of
genomes is considered and represented as a (pangeome) graph.
In this case, my main contribution has been the introduction of a formal notion of the computational
problem of pangenome graph construction, which in turn allows to analyze
existing and new algorithms for the problem.
Moreover, I studied efficient and practical
sequence-to-graph alignment algorithms that fully exploit the topology of the
graphs to infer recombination events --- this problem was not previously
attacked with approaches based on sequence alignment.


\textbf{Journal papers}

\cite{baaijensComputationalGraphPangenomics2022}
\cite{DBLP:journals/bioinformatics/DentiPPCVRB21}
\cite{DBLP:journals/ipl/BonizzoniVDP10}
\cite{DBLP:journals/tcbb/BonizzoniVDFRV07}
\cite{DBLP:journals/informs/JustV04}
\cite{DBLP:journals/dam/BonizzoniVM01}
\cite{DBLP:journals/tcs/BonizzoniV01}

\textbf{Software}

\begin{itemize}
\item
      RecGraph \url{https://github.com/AlgoLab/RecGraph}
\end{itemize}

\textbf{Collaborations}

\begin{itemize}
\item
      Universitaet Bielefeld (Alexander Schoenhuth)
\item
      University of California at Santa Cruz (Jouni Siren)
\end{itemize}

\subsection{Phylogeny reconstruction and
comparison}\label{phylogeny-reconstruction-and-comparison}

Reconstructing phylogenies is another problem that has great relevance
in Computational Biology, as a phylogeny is an intuitive representation
of a common evolutionary history of a set of extant species. In this
setting I have studied the quartet-based reconstruction technique (a
quartet is the optimal solution over four species), developing new
algorithms to clean some of the error that inherent in the use of such
technique.

More recently, I have studied the algorithmic implications of new models of
evolution, especially for tumor and viral phylogeny.
In fact, most of the literature focuses on a very restrictive model of evolution
where mutations can only be acquired.
Cancer evolution is too complex for that model, in fact losses of entire regions
of the genome is common.
Unfortunately, this makes the computational problem much harder, since the
solution space explodes.
My contributions on these aspects spans from mathematical exploration of
combinatorial properties of phylogenies to efficient and practical algorithms to
infer phylogenies, leading their design, development, and analysis.


The phylogeny comparison problem is fundamental when you have to compare
the results of various experiments on the same set of species. I have
extensively studied the problem In (Amir e Keselman, SIAM J. Comp. 1997)
some formulations of the problem have been introduced. My initial contributions
have been on the approximability of the maximum common isomorphic
subtree, and on efficient algorithms to reconcile gene and species trees.
Most recently, I have shifted my focus on practical similarity measures between
tumor phylogenies.
In fact, there are several tools for inferring tumor phylogenies, but the
research community needs a measure to understand which of the phylogenies are
more similar then others.


\textbf{Journal papers}

\cite{DBLP:journals/jcb/AliCLVP21}
\cite{DBLP:journals/bioinformatics/CiccolellaRGPSB21}
\cite{DBLP:journals/bioinformatics/CiccolellaBDBPV21}
\cite{DBLP:journals/bmcbi/CiccolellaGPVHB20}
\cite{DBLP:journals/tcbb/BonizzoniCVS19}
\cite{DBLP:journals/tcs/BonizzoniCVRT17}
\cite{DBLP:journals/fuin/BonizzoniCVRT17}
\cite{bonizzoniExplainingEvolutionConstrained2014}
\cite{DBLP:journals/tcs/BonizzoniVD05}
\cite{DBLP:journals/bioinformatics/VedovaW02}
\cite{DBLP:journals/ijfcs/BonizzoniVM00}


\textbf{Software}

\begin{itemize}
\item
     SASC \url{https://github.com/sciccolella/sasc}
\item
      MP3 \url{https://github.com/AlgoLab/mp3treesim}
\item
      ggpf \url{https://github.com/AlgoLab/gppf}
\item
      ggps \url{https://github.com/AlgoLab/gpps}
\end{itemize}

\textbf{Collaborations}

\begin{itemize}
\item
      Georgia State University (Murray Patterson)
\item
      Weill Cornell Medicine (Iman Hajirasouliha)
\end{itemize}

\subsection{Clustering}\label{clustering}

The problem of classifying data in similar sets is one of the most
important problems in Computer Science; it is common to have a
similarity measure between pairs of elements and to aim at computing a
partition of the elements so that elements in a common sets are similar
while elements in different sets are not similar. In this field
I have studied the correlation clustering
problem on weighted graphs, which has the important property that the number of
clusters is not fixed a priori, but depends on the dataset analyzed.
My contribution has been to develop the connections with another version of
clustering, called consensus clustering, showing that an
interesting restriction of the problem is NP-complete and providing two
polynomial-time approximation schemes for a different formulation of the
problem.

Moreover I have studied a different clustering problem, closely related
to the analysis of microarray data. In this case, data are represented
as vectors on \(\{0,1,N\}\) alphabet, where \(N\) stands for missing or
undecided data. His contributions consist of proving that some
restrictions of the problem are APX-hard and designing a polynomial-time
constant-factor approximation algorithm. Moreover, I have designed an
efficient algorithm for a different restriction of the problem.

Another research topic related to Clustering is the \(k\)-anonymity problem,
where we want to cluster the rows of a matrix so that each cluster has identical
entries, except for \(k\) columns, where I have designed some algorithms that
are efficient for small values of \(k\).
This formulation has originally appeared in data privacy, but I am investigating
its applicability to pangenome privacy.


Tumor phylogeny inference methods are especially computationally expensive.
For this reason, I have developed some clustering techniques that are tailored
for clustering single-cell tumoral data.
These new approaches are based on community detection, and I have proved
experimentally that they result in faster and more precise phylogenies.


\textbf{Journal papers}

\cite{chourasiaReads2VecEfficientEmbedding2023}
\cite{10.1093/gigascience/giac119}
\cite{DBLP:journals/titb/CiccolellaPBV21}
\cite{DBLP:journals/jco/BonizzoniVDP13}
\cite{DBLP:journals/tcs/BonizzoniVD12}
\cite{DBLP:journals/jco/BonizzoniVD11}
\cite{DBLP:journals/algorithmica/BonizzoniVDM10}
\cite{DBLP:journals/jcss/BonizzoniVDJ08}

\textbf{Software}

\begin{itemize}
\item
      plastic \url{https://github.com/plastic-phy/plastic}
\item
      CouGaR-g \url{https://github.com/AlgoLab/CouGaR-g}
\item
      MALVIRUS \url{https://algolab.github.io/MALVIRUS/}
\item
      celluloid \url{https://github.com/AlgoLab/celluloid}
\end{itemize}

\textbf{Collaborations}

\begin{itemize}
      \item
      Georgia State University (Murray Patterson)
\end{itemize}

\subsection{Alternative Splicing}\label{splicing}

Alternative splicing is the biological mechanism that allows a gene to encode
and produce more than one protein and it is correlated to the onset of several diseases.
In this field I have focused on efficient algorithms for detecting novel (that
is not previously known) alternative splicing events.

I have supervised the development of two approaches to infer novel alternative
splicing events, based on the alignment of reads against a transcriptome.
In the second approach, the alignment is against a graph transcriptome which has
been one of the very first papers introducing graph-based representation of a
set of transcripts.
I have supervised the algorithm design and the experimental analysis.

The sheer amount of data requires some efficient filtering.
I have developed a fast and accurate filter, based on Bloom filters, that is
able to extract from a sample on the much smaller reads that originate from a
given gene, allowing to use a more precise, albeit slower, downstream tool.
I have developed the algorithm and supervised the experimental analysis.


\textbf{Software}

\begin{itemize}
\item
      PIntron \url{https://github.com/AlgoLab/PIntron}
\item
      ASGAL \url{https://github.com/AlgoLab/galig}
\item
      Shark
\end{itemize}



\textbf{Main collaborations}

\begin{itemize}
\item
      CNR (Graziano Pesole)
\end{itemize}

\textbf{Journal papers}

\cite{DBLP:journals/bioinformatics/DentiPPCVRB21}
\cite{DBLP:journals/bmcbi/DentiRBVPB18} \cite{DBLP:journals/jcb/BerettaBVPR14} \cite{DBLP:journals/bmcbi/PirolaRPPVB12}


\subsection{DNA microarray design}\label{dna-microarray-design}

The introduction of DNA microarray have greatly increased the throughput
of experimental data in Molecular Biology. Such technology (Drmanac et.
al.~Science 1991) has given relevance to some computational problems on
the optimal microarray synthesis or on the experimental data analysis.

More precisely, classifying microbial communities can be performed only
exploiting microarrays, as the microbial external aspect is hard to
study, due to their extremely small size. A fundamental computational
problem in this field is computing the minimum set of substrings that
are able to distinguish a set of strings (Probe Selection). The
importance of this problem is due to the fact that it formalizes the
search for the cheapest experiment obtaining the desired result.

I have proposed, implemented and analyzed some algorithms, while
supervising the use of the implementation on some biological data that
were previously impossible to analyze. This effort has led to a new
protocol for microbial communities analysis.

\textbf{Journal  papers}

\cite{DBLP:journals/bmcbi/RueggerVJB11}
\cite{valinskyAnalysisBacterialCommunity2002}
\cite{valinskyOligonucleotideFingerprintingRRNA2002}
\cite{bornemanProbeSelectionAlgorithms2001}

\textbf{Software}

\begin{itemize}
\item
      ProbeSelection
\end{itemize}


\textbf{Main collaborations}

\begin{itemize}
\item
      University of California at Riverside (Tao Jiang, James Borneman)
\end{itemize}

\subsection{Haplotyping}\label{haplotyping}

Several species, including human beings, are diploid, that is each
chromosome consists of two distinct copies called haplotypes. Current
technological limitations do not allow to cheaply compute those
haplotypes, but only genotypes (that is the two nucleotides that are in
the same position in those haplotypes). Since it is important to know
the actual haplotypes, a number of related computational problems have
been recently introduced; those problems compose the field of
haplotyping.

I have designed and analyzed an algorithm to complete haplotypes on
incomplete data and under the coalescent model (such models forbids some
otherwise possible recombinations), where the objective function is the
entropy of the solution. The algorithm belongs to the class of
Kernighan-Lin heuristics and it has been empirically and favorably
compared to the greedy algorithm that was previously routinely employed.

Moreover I have studied the xor-genotyping problem, where the input data
contains even less information that in most other haplotyping problems,
as it is known only the positions where the two haplotypes differ, and
not their contents. This formulation is a faithful model of the results
that can be obtained with a recent and economically viable technology.
I have obtained some preliminary results on the computational complexity
of the problem, and he has designed some efficient algorithms.

Another problem that I have studied is single-individual haplotype assembly, where we want to
determine which variants a person actually has.
My contribution on this problem has been the development of a software tool
(HapChat) and the supervision of its experimental analysis.
This approach has been incorporated into one of the most widely used tools for
this problem.
Most recently, we have extended this idea to viral data, where we want to
quickly and effectively identify the viral strain in a sample.
In this case I have supervised the entire work.


\textbf{Journal papers}

\cite{ciccolellaMALVIRUSIntegratedApplication2022}
\cite{DBLP:journals/bmcbi/BerettaPZVB18}
\cite{DBLP:journals/tcbb/PirolaVBSB12}
\cite{DBLP:journals/tcbb/BonizzoniVDPR10}
\cite{DBLP:journals/ijbra/BonizzoniVDM05}
\cite{DBLP:journals/jcst/BonizzoniVDL03}

\textbf{Software}

\begin{itemize}
\item
      HapChat \url{https://github.com/AlgoLab/HapCHAT}
\end{itemize}

\subsection{Graph Algorithms}\label{graph-algorithms}

Graph theory is one of the most
important research fields that are common to Computer Science and
Discrete Mathematics, as graphs are a mathematical device that is
suitable for natural modeling of various real-world problems. One of the
techniques that has been widely employed for designing efficient graph
algorithm consists of decomposing the graph and then solving the problem
on the smaller parts for finally recombining the partial solutions.

I have developed some efficient algorithms for
computing the modular decomposition on hypergraphs and k-structures.


\textbf{Journal papers}

\cite{DBLP:journals/jal/BonizzoniV99}


\subsection{Text Algorithms}\label{bwt}

The sheer size of genomic data makes important to develop data structures to
represent with a small amount of memory while querying them efficiently.
The main such data structure is the Burrows-Wheeler Transform (BWT), and I have
developed some algorithms to efficiently compute the BWT together with another
auxiliary data structure (the Longest Common Prefix array) and for building the
string graph (a data structure that represents how some strings overlap with
each other).
Moreover, I have supervised the implementation and the experimental analysis of
those algorithms.


\textbf{Journal papers}

\cite{DBLP:journals/tcs/BonizzoniVPPR21}
\cite{DBLP:journals/jcb/BonizzoniVPPR19}
\cite{DBLP:journals/quanbio/RizziBPPPVB19}
\cite{DBLP:journals/algorithmica/BonizzoniVPPR17}
\cite{DBLP:journals/jcb/BonizzoniVPPR17}
\cite{DBLP:journals/jcb/BonizzoniVPPR16}


\textbf{Software}

\begin{itemize}
\item
      bwt-lcp-parallel \url{https://github.com/AlgoLab/bwt-lcp-parallel}
\item
      bwt-lcp-em \url{https://github.com/AlgoLab/bwt-lcp-em}
\item
      FSG \url{https://github.com/AlgoLab/FastStringGraph}
\item
      LSG \url{https://github.com/AlgoLab/LightStringGraph}
\end{itemize}


  \section{Research and Technology Transfer}

  \subsection{Research and consulting contracts}


\cvitem{2003--2004}{%
  Consultant on the computational aspects of the project INTERREG IIIB (2000--2006) W.E.S.T. WOMEN EAST SMUGGLING
  TRAFFICKING (WP.2.2).
  \textbf{Proponent: Fondazione Ismu -- Iniziative e
  Studi sulla Multietnicità}.
  I have been the only computer scientist involved in the project.
  }

  \cvitem{2005}{%
  Consultant on the computational aspects of the project
  ``Indagine Finalizzata all'Analisi degli Effetti Prodotti dai
  Processi di Regolarizzazione dei Lavoratori Extracomunitari, con
  Particolare Riferimento al Mercato del Lavoro e all'Integrazione Sociale
  nelle Regioni Ob. 1'', funded under Misura I.2 FESR
  ``Adeguamento del Sistema di Controllo Tecnologico del Territorio'',
  PON Sicurezza 2000/2006.
  \textbf{Proponent: Fondazione Ismu -- Iniziative e
  Studi sulla Multietnicità}.
  I have been the only computer scientist involved in the project.
  }

\section{Publications}


\nociteJ{*}
\bibliographystyleJ{unsrt}
\bibliographyJ{article}

\nociteC{*}
\bibliographystyleC{unsrt}
\bibliographyC{conference}

\nociteI{*}
\bibliographystyleI{unsrt}
\bibliographyI{inbook}


\nociteP{*}
\bibliographystyleP{unsrt}
\bibliographyP{preprint}

Milano, \today\\

Gianluca Della Vedova

\end{document}
%
% reftex-bibliography-commands: ('bibliographystylearticle, 'bibliographystyleconference, 'bibliographystyleinbook, 'bibliographystyleP)
%
