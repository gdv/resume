% Options for packages loaded elsewhere
\PassOptionsToPackage{unicode}{hyperref}
\PassOptionsToPackage{hyphens}{url}
%
\documentclass[
]{article}
\usepackage{amsmath,amssymb}
\usepackage{lmodern}
\usepackage{iftex}
\ifPDFTeX
  \usepackage[T1]{fontenc}
  \usepackage[utf8]{inputenc}
  \usepackage{textcomp} % provide euro and other symbols
\else % if luatex or xetex
  \usepackage{unicode-math}
  \defaultfontfeatures{Scale=MatchLowercase}
  \defaultfontfeatures[\rmfamily]{Ligatures=TeX,Scale=1}
\fi
% Use upquote if available, for straight quotes in verbatim environments
\IfFileExists{upquote.sty}{\usepackage{upquote}}{}
\IfFileExists{microtype.sty}{% use microtype if available
  \usepackage[]{microtype}
  \UseMicrotypeSet[protrusion]{basicmath} % disable protrusion for tt fonts
}{}
\makeatletter
\@ifundefined{KOMAClassName}{% if non-KOMA class
  \IfFileExists{parskip.sty}{%
    \usepackage{parskip}
  }{% else
    \setlength{\parindent}{0pt}
    \setlength{\parskip}{6pt plus 2pt minus 1pt}}
}{% if KOMA class
  \KOMAoptions{parskip=half}}
\makeatother
\usepackage{xcolor}
\IfFileExists{xurl.sty}{\usepackage{xurl}}{} % add URL line breaks if available
\IfFileExists{bookmark.sty}{\usepackage{bookmark}}{\usepackage{hyperref}}
\hypersetup{
  pdftitle={Della Vedova Gianluca},
  hidelinks,
  pdfcreator={LaTeX via pandoc}}
\urlstyle{same} % disable monospaced font for URLs
\setlength{\emergencystretch}{3em} % prevent overfull lines
\providecommand{\tightlist}{%
  \setlength{\itemsep}{0pt}\setlength{\parskip}{0pt}}
\setcounter{secnumdepth}{-\maxdimen} % remove section numbering
\ifLuaTeX
  \usepackage{selnolig}  % disable illegal ligatures
\fi

\title{Della Vedova Gianluca}
\author{}
\date{}

\begin{document}
\maketitle

\begin{quote}
Professore Associato Confermato (SSD INF/01 - Informatica)
\end{quote}

\begin{quote}
Dipartimento di Informatica, Sistemistica e Comunicazione • Università
degli Studi di Milano -- Bicocca
\end{quote}

\begin{center}\rule{0.5\linewidth}{0.5pt}\end{center}

\begin{description}
\tightlist
\item[10/2012-oggi]
Professore Associato, Dip. Informatica, Sistemistica e Comunicazione,
Università degli Studi di Milano -- Bicocca.
\item[10/2005-09/2012]
Professore Associato, Facoltà di Scienze Statistiche, Università degli
Studi di Milano -- Bicocca.
\item[05/2001-09/2005]
Ricercatore Universitario, Facoltà di Scienze Statistiche, Università
degli Studi di Milano -- Bicocca.
\end{description}

\hypertarget{formazione}{%
\subsection{Formazione}\label{formazione}}

\begin{description}
\tightlist
\item[2001]
\textbf{Dottorato di Ricerca in Informatica}; Università degli Studi di
Milano
\item[1995]
\textbf{Laurea in Scienze dell'Informazione}; Università degli Studi di
Milano.
\end{description}

\hypertarget{ricerca}{%
\subsection{Ricerca}\label{ricerca}}

L'attività di ricerca si concentra sul disegno e l'analisi di algoritmi
in bioinformatica. L'obiettivo delle attività di ricerca è lo sviluppo
di algoritmi che siano efficienti sia dal punto di vista teorico che
pratico. Di conseguenza le tecniche utilizzate sono sia di natura
teorica che sperimentale. In particolare sono state affrontate le
tematiche del confronto di sequenze biologiche, la ricostruzione di
alberi evolutivi e di evoluzioni tumorali, l'inferenza di aplotipi, il
riconoscimento di eventi di splicing alternativi.

Altri ambiti di ricerca sono algoritmi su grafi, clustering e data
privacy.

Ha pubblicato oltre 30 articoli su riviste scientifiche internazionali.
H-index: 16. Citazioni: 1381 (fonte
\href{https://scholar.google.com/citations?user=0gaIFokAAAAJ\&hl=en\&oi=ao}{Google
Scholar}).

Responsabile del progetto Grandi Attrezzature ``Laboratorio Virtuale
Statistico-Territoriale'' e di diversi Fondi di Ateneo.

\hypertarget{didattica}{%
\subsection{Didattica}\label{didattica}}

Dal 2001 a oggi ha insegnato nei corsi di Laurea in Statistica;
Statistica e Gestione delle Informazioni; Scienze Statistiche ed
Economiche; Informatica.

Ha insegnato nei corsi di Laurea specialistica/magistrale in
Biostatistica; Scienze Statistiche ed Economiche; Data Science, oltre
che nei master in Bioinformatica; Master in Amministrazione di sistemi
diagnostici per immagini; Master in Bioinformatica.

Membro del collegio docenti e vice-coordinatore del Dottorato di ricerca
in Informatica.

\hypertarget{servizio}{%
\subsection{Servizio}\label{servizio}}

\begin{itemize}
\tightlist
\item
  E' stato membro del Editorial Board di ``Advances in Bioinformatics''.
\item
  Editor di special issue di ``Journal of Computer Science and
  Technology''
\item
  Membro di comitato di programma delle conferenze scientifiche CiE 2013
  2019 2020, SPIRE 2017, CPM 2019, WABI 2020.
\item
  Membro di comitato di programma dei workshop ``Workshop on Graph
  Assembly Algorithms for omics data'', ``Workshop on Algorithms in
  Molecular Biology''
\item
  Membro del Comitato direttivo del Centro di Produzione Multimediale di
  Ateneo
\item
  Supervisore di 3 tesi di dottorato e di oltre 10 tesi di laurea.
\item
  E' membro della Commissione Paritetica del Dipartimento di
  Informatica, Sistemistica e Comunicazione
\item
  E' Assicuratore di qualità del corso di laurea magistrale in Data
  Science.
\item
  E' stato delegato per l'e-learning del Preside della Facoltà di
  Scienze Statistiche.
\item
  E' stato rappresentante dell'Area informatica all'interno della
  Facoltà di Scienze Statistiche.
\item
  E' stato membro del comitato incaricato di progettare il corso di
  laurea triennale \emph{Statistica e Gestione delle Informazioni}.
\item
  E' stato il docente responsabile dei laboratori informatici della
  Facoltà di Scienze Statistiche.
\item
  E' stato responsabile del corso 167388 ``Laboratorio Complementare di
  Informatica per Statistici'', parte del progetto FSE 156165 ``Progetto
  Quadro Università degli Studi di Milano-Bicocca'', interamente
  finanziato da EU.
\item
  E' stato il responsabile per la Facoltà di Scienze Statistiche del
  corso ``Information Technology For Problem Solving (IT4PS)'',
  organizzato dalla Fondazione CRUI.
\end{itemize}

Milano, 17/02/2020

\begin{center}\rule{0.5\linewidth}{0.5pt}\end{center}

\begin{quote}
\href{mailto:gianluca.dellavedova@unimib.it}{\nolinkurl{gianluca.dellavedova@unimib.it}}
• \url{https://gianluca.dellavedova.org}
\end{quote}

\begin{quote}
\href{https://orcid.org/0000-0001-5584-3089}{orcid 0000-0001-5584-3089}
\end{quote}

\begin{center}\rule{0.5\linewidth}{0.5pt}\end{center}

\end{document}
