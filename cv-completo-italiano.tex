\documentclass[11pt,a4paper,roman]{moderncv}        % possible options include font size ('10pt', '11pt' and '12pt'), paper size ('a4paper', 'letterpaper', 'a5paper', 'legalpaper', 'executivepaper' and 'landscape') and font family ('sans' and 'roman')
\usepackage[italian]{babel}
% moderncv themes
\moderncvstyle{banking}                             % style options are 'casual' (default), 'classic', 'oldstyle' and 'banking'
\moderncvcolor{blue}                               % color options 'blue' (default), 'orange', 'green', 'red', 'purple', 'grey' and 'black'
%\renewcommand{\familydefault}{\sfdefault}         % to set the default font; use '\sfdefault' for the default sans serif font, '\rmdefault' for the default roman one, or any tex font name
%\nopagenumbers{}                                  % uncomment to suppress automatic page numbering for CVs longer than one page

% character encoding
\usepackage[utf8]{inputenc}                       % if you are not using xelatex ou lualatex, replace by the encoding you are using

\usepackage{verbatim}


% adjust the page margins
\usepackage[scale=0.75]{geometry}
%\setlength{\hintscolumnwidth}{3cm}                % if you want to change the width of the column with the dates
%\setlength{\makecvtitlenamewidth}{10cm}           % for the 'classic' style, if you want to force the width allocated to your name and avoid line breaks. be careful though, the length is normally calculated to avoid any overlap with your personal info; use this at your own typographical risks...

% personal data
\name{Gianluca}{Della Vedova}
\email{gianluca.dellavedova@unimib.it}
\homepage{https://www.unimib.it/gianluca-della-vedova}
\social[github]{gdv}
%\quote{Some quote no}                                 % optional, remove / comment the line if not wanted

% to show numerical labels in the bibliography (default is to show no labels); only useful if you make citations in your resume
%\makeatletter
%\renewcommand*{\bibliographyitemlabel}{\@biblabel{\arabic{enumiv}}}
%\makeatother
%\renewcommand*{\bibliographyitemlabel}{[\arabic{enumiv}]}% CONSIDER REPLACING THE ABOVE BY THIS

\title{CV}                               % optional, remove / comment
% bibliography with mutiple entries
\usepackage{multibib}
\newcites{inbook,article,conference}{{Capitoli di libro},{Articoli in riviste},{In atti di convegno}}
%----------------------------------------------------------------------------------
%            content
%----------------------------------------------------------------------------------
\begin{document}
%\begin{CJK*}{UTF8}{gbsn}                          % to typeset your resume in Chinese using CJK
%-----       resume       ---------------------------------------------------------
\makecvtitle

\section{Attuale posizione}

\noindent
  Professore Associato-Confermato (SSD INF/01 -- Informatica)\\
  Dipartimento di Informatica, Sistemistica e Comunicazione\\
  Università degli Studi di Milano -- Bicocca

\section{Ruoli}\label{ruoli}

  \cventry{10/2012--oggi}{Dip. Informatica, Sistemistica e Comunicazioni}{Professore Associato}{Università degli Studi di Milano -- Bicocca}{}{}
  \cventry{10/2005--09/2012}{Facoltà di Scienze Statistiche}{Professore Associato}{Università degli Studi di Milano -- Bicocca}{}{}
  \cventry{05/2001--09/2005}{Facoltà di Scienze Statistiche}{Ricercatore}{Università degli Studi di Milano -- Bicocca}{}{}

\section{Abilitazione}\label{abilitazione}

  \cventry{08/2018--oggi}{}{Abilitazione a Professore di I fascia, settore concorsuale 01/B1}{}{}{}

\section{Titoli di studio}

  \cventry{1996--2001}{}{Dottorato di Ricerca in Informatica}{Università degli Studi di
  Milano}{}{Advisors: Tao
  Jiang (UC Riverside), Giancarlo Mauri (Univ. Milano -- Bicocca),
  Paola Bonizzoni (Univ. Milano -- Bicocca)}{Tesi: ``Multiple Sequence Alignment and Phylogenetic Reconstruction:
  Theory and Methods in Biological Data Analysis''}

  \cventry{1990--1995}{}{Laurea in Scienze dell'Informazione}{Università degli Studi di Milano}{}{Tesi: Algoritmi sequenziali e paralleli per la decomposizione di grafi}  %

\section{Progetti di ricerca}


\subsection{Responsabile}


  \cvitem{2020--2023}{%
  H2020-MSCA-RISE-2019 (importo Univ. Milano -- Bicocca 197 800€) nell'ambito
  dell'azione Research Innovation Staff Exchange. Pan-genome Graph
  Algorithms and Data Integration (PANGAIA). In questo progetto, con
  coordinatore del progetto interno al mio laboratorio di ricerca, sono
  stato fra i principali estensori della proposta e sono il
  \textbf{responsabile del WP5} Communication and Dissemination. Il
  progetto prevede la collaborazione fra 7 istituzioni beneficiarie (4
  Università, 1 ente di ricerca e 2 aziende) di 6 nazioni europee e 4
  partner in USA, Canada e Giappone.
  }
  \cvitem{2021--2024}{%
  H2020-MSCA-ITN (importo Univ. Milano -- Bicocca 261499,68€) nell'ambito
  dell'azione Innovative Training Network. ALgorithms for PAngenome
  Computational Analysis (ALPACA). In questo progetto ho la responsabilità
  di \textbf{co-supervisionare un dottorando} nel periodo 2021--2024. Il
  progetto vede 13 beneficiari (7 università, 5 enti di ricerca, 1
  azienda) e 10 partner europei.
  }
  \cvitem{2013--2016}{%
  Fondazione Cariplo 2013 Modulation of anti-cancer immune response by
  regulatory non-coding RNAs. Sono stato \textbf{responsabile del WP
  bioinformatico}. Il progetto prevedeva due unità: l'Istituto Nazionale
  di Genetica Molecolare e l'Univ. Milano -- Bicocca. Il progetto mi ha
  portato ad essere \textbf{responsabile Scientifico di due assegni di
  ricerca} annuali e ho co-supervisionato le attività di un terzo assegno
  di ricerca annuale.
  }
  \cvitem{2011--2014}{%
  MIUR/Regione Lombardia 2011 (importo Univ. Milano -- Bicocca 199 991€).
  Piattaforma di Analisi TRaslazionale Integrata. In questo progetto sono
  stato il \textbf{responsabile di tutti gli aspetti bioinformatici}. Il
  progetto ha portato all'attivazione di un assegno di ricerca annuale
  (costo 23.000€) e 3 contratti di collaborazioni (ognuno di importo lordo
  al collaboratore fra 10.000€ e 12.000€). Ho co-supervisionato le
  attività relative all'assegno di ricerca, e sono stato unico
  responsabile scientifico delle attività relative ai contratti di
  collaborazione.
  }
  \cvitem{2016}{%
  Fondo di Ateneo 2016 (12 490€). Modelli computazionali e algoritmi:
  aspetti teorici e sperimentali, con applicazioni alla Bioinformatica.
  \textbf{Responsabile} del progetto.
  }
  \cvitem{2015}{%
  Fondo di Ateneo 2015 (10 980€). Algoritmi combinatori e modelli di
  calcolo: aspetti teorici e applicazioni in Bioinformatica.
  \textbf{Responsabile} del progetto.
  }
  \cvitem{2014}{%
  Fondo di Ateneo 2014 (12 186€). Algoritmi e modelli computazionali:
  aspetti teorici e applicazioni nelle scienze della vita.
  \textbf{Responsabile} del progetto.
  }
  \cvitem{2013}{%
  Fondo di Ateneo 2013 (9 337€). Metodi algoritmici e modelli: aspetti
  teorici e applicazioni in bioinformatica. \textbf{Responsabile} del
  progetto.
  }
  \cvitem{2011}{%
  Fondo di Ateneo 2011 (4 055€). Tecniche algoritmiche avanzate in
  Biologia Computazionale. \textbf{Responsabile} del progetto.
  }
  \cvitem{2006}{%
  Grandi Attrezzature 2006 (40000€). Laboratorio Virtuale
  Statistico-Territoriale. Questo progetto, condiviso fra il Dipartimento
  di Statistica e il Dipartimento di Sociologia, ha portato all'acquisto
  di 2 server per la fornitura di servizi di ricerca e didattica in ambiti
  di Sociologia del territorio e Statistica computazionale.
  \textbf{Responsabile} del progetto.
  }
  \cvitem{2005--2008}{%
  MIUR/PRIN 2005, Potenzialità e ottimizzazione delle banche dati
  automatizzate in epidemiologia. In questo progetto sono stato il
  \textbf{responsabile di tutti gli aspetti bioinformatici}.
  }

\subsection{Partecipante}


\cvitem{2013--2016}{%
  Regione Lombardia. SPAC3 --- Servizi smart della nuova Pubblica
  amministrazione per la Citizen-Centricity in cloud.}
  \cvitem{2011--2014}{%
  MIUR/PRIN 2011
  Automi e Linguaggi Formali: Aspetti Matematici e Applicativi}
  \cvitem{2003}{%
  MIUR/FIRB 2003. Bioinformatica per la Genomica e la Proteomica.}
  \cvitem{2000--2001}{%
  NSF CCR-9988353, ITR-0085910.}
  \cvitem{1999--2001}{%
  MURST COFIN 98 ``Bioinformatica e ricerca genomica''.}
  \cvitem{1994--1995}{%
  MURST 40\% ``Algoritmi e strutture di calcolo''.}
  \cvitem{1994--1995}{%
  ESPRIT-BRA ASMICS 2 n.~6317.
  }

  \section{Ricerca}



  Ho pubblicato oltre 30 articoli su rivista scientifica con più di 1500
  citazioni e h-index 17 (secondo Google Scholar). In particolare, 7
  lavori hanno superato le 100 citazioni (fonte Google Scholar). Inoltre
  ho lavorato con 166 coautori (fonte Scopus).

  Dal 2016 al 2019 sono stato responsabile del Laboratorio di Ricerca
  ``AlgoLab --- Experimental Algorithmics Lab'', DIpartimento di
  Informatica, Sistemistica e Comunicazione, Università di Milano -- Bicocca.
  Il laboratorio di ricerca si è caratterizzata per numerose
  collaborazioni internazionali. Il laboratorio si occupa del disegno di
  algoritmi, della loro implementazione e dell'analisi sperimentale su
  dataset di grandi dimensioni, pertanto l'ambito di ricerca riguarda sia
  aspetti metodologici che sperimentali: ciò ha portato alla realizzazione
  di vari strumenti software per risolvere vari problemi in
  bioinformatica.

  La mia ricerca si è sempre focalizzata sullo sviluppo di algoritmi
  combinatori in Bioinformatica, con una forte componente fondazionale
  basata sullo studio delle proprietà formali dei problemi computazionali
  e passando per l'implementazione degli approcci proposti ed una
  validazione degli stessi sia da un punto di vista teorico che
  sperimentale.

  Le principali tematiche investigate sono state:

  \textbf{Confronto di sequenze}

  Il confronto di sequenze è uno dei problemi fondamentali in
  Bioinformatica, in quanto sequenze simili corrispondono a parti di
  genoma con funzionalità simili. In questo campo ha ottenuto importanti
  risultati sulla complessità computazionale di alcune formulazioni del
  problema, disegnando algoritmi efficienti per il calcolo di soluzioni
  approssimate.

  Le attività in questo ambito sono state svolte anche nell'ambito delle
  collaborazioni scientifiche con il Winfried Just (Dept. Mathematics,
  Ohio Univ.), Stéphane Vialette (LRI, Univ. Paris-Sud), Guillame Fertin
  (LINA, Univ. Nantes).

  \textbf{Ricostruzione storie evolutive}

  Il problema di ricostruire la storia evolutiva a partire da dati
  genomici di specie o cellule esistenti è stato uno dei principali ambiti
  della mia attività di ricerca. Inizialmente mi sono focalizzato su
  problematiche classiche di confronto di alberi evolutivi, ottenendo
  risultati sulla complessità di approssimazione, per poi descrivere
  algoritmi per la riconciliazione di alberi di geni e di specie (per
  individuare una storia evolutiva comune che sia una interpretazione
  ammissibile di alberi apparentemente incompatibili). Più recente ho
  contribuito a disegnare approcci efficienti per il problema della
  ricostruzione di storie evolutive tumorali, anche supervisionando le
  attività del gruppo di ricerca, oltre all'implementazione e l'analisi
  sperimentali di tali approcci.

  Questa tematica di ricerca è stata svolta anche nell'ambito delle
  collaborazioni internazionali con Harold Todd Wareham (Dept. Computer
  Science, Memorial Univ. Newfoundlands), Tao Jiang (Dept. Computer
  Science, Univ. California at Riverside), Gabriella Trucco (Univ.
  Milano), Jesper Jansson (The Hong Kong Polytechnic University), Iman
  Hajirasouliha (Weill Cornell Medical College, New York). Responsabile
  della collaborazione scientifica con il prof. Vladimir Filipovic, Univ.
  Belgrado, che è in sabbatico presso l'Univ. Milano -- Bicocca dal 20/1/18.
  La collaborazione riguarda lo sviluppo di metaeuristiche in
  Bioinformatica, ed in particolare per la ricostruzione di evoluzioni
  tumorali.

  \textbf{Ricostruzioni di aplotipi}

  Diverse specie, incluso l'uomo, sono diploidi: ogni cromosoma è presente
  in due copie distinte detti aplotipi. Le tecnologie attuali riescono
  solo con difficoltà a determinare l'aplotipo di provenienza. In questo
  ambito, approcci informatici sono essenziali per riuscere ad ottenere i
  singoli aplotipi ad un costo accettabile. Dopo avere ottenuti alcuni
  risultati sulla complessità computazionale del problema, mi sono
  dedicato al disegno di approcci euristici efficienti, seguendo anche gli
  aspetti di implementazione e di analisi sperimentale. Il codice prodotto
  è stato incorporato in uno dei principali programmi utilizzati dalla
  comunità scientifica
  (https://whatshap.readthedocs.io/en/latest/howtocite.html).

  Le attività in questo ambito sono state svolte anche nell'ambito delle
  collaborazioni scientifiche con Tao Jiang (Univ. California at
  Riverside), Alessandra Stella (CNR), Tobias Marschall (Heinrich Heine
  Univ., Düsseldorf), Romeo Rizzi (Univ. Verona).

  \textbf{Splicing alternativo}

  Lo splicing alternativo è il fenomeno biologico che permette ad un
  singolo gene di esprime più proteine. Dal punto di vista computazionale,
  il problema principale è che non abbiamo in input l'intera sequenza che
  viene tradotta in proteine (detta trascritto), ma solo sue porzioni.
  Pertanto diventa necessario esaminare queste porzioni per costruire
  tutti i trascritti, tenendo conto che presentano ripetizioni ed errori.
  In questo ambito ho contributito a disegnare approcci efficienti,
  seguendo anche gli aspetti di implementazione e di analisi sperimentale.

  Le attività in questo ambito sono state svolte anche nell'ambito delle
  collaborazioni scientifiche con Graziano Pesole (CNR e Head of Node di
  Elixir Italy IIB), Ernesto Picardi (Univ. Bari).

  \textbf{Algoritmi su Grafi}

  Gli aspetti combinatoriali di teoria dei grafi sono un ambito dove ho
  contribuito allo sviluppo di algoritmi efficienti, principalmente per il
  problema della decomposizione modulare, che è una tecnica che permette
  il disegno di algoritmi efficienti per diversi problemi. Più
  recentemente ho approfondito aspetti di teoria dei grafi, riuscendo a
  sfruttarli in diversi ambiti della Bioinformatica, sia per ottenere
  modelli computazionali più aderenti alla realtà biologica, sia per
  ottenere approcci efficienti. Ciò è stato fondamentale nell'attività
  inerente al settore innovativo della Pangenomica Computazionali, dove si
  intendono studiare insiemi di genomi tramite strutture a grafo. Tale
  ambito è l'oggetto dei progetti di ricerca europei PANGAIA (MSCA RISE
  2019) e ALPACA (MSCA ITN 2020) indicati in precedenza.

  Le attività in questo ambito sono state svolte anche nell'ambito delle
  collaborazioni scientifiche con Alexander Schönhuth (Univ. Bielefeld).

  \textbf{Clustering}

  Il clustering è uno dei problemi più studiati in Informatica, anche a
  causa delle sue molteplici varianti ed applicazioni pratiche. Oltre ad
  avere contribuito all'avanzamento delle conoscenze relative alla
  complessità computazionale del problema del confronto di
  cluterizzazioni, ho disegnato algoritmi efficienti per la
  clusterizzazione di fingerprint, problema che trova la sua motivazione
  nello studio di comunità microbiche. Inoltre ho contribuito a risultati
  sulla complessità computazionale sul problema di anonimizzare tabelle
  tramite l'omissione di dati, un problema che trova la sua motivazione
  nell'ambito della data privacy.


  \section{Didattica}


  \subsection{Attività didattiche}

  La mia attività didattica è iniziata come Ricercatore Universitario
  presso la Facoltà di Scienze Statistiche nel 2001. Poichè in Facoltà non
  erano presenti altri docenti dell'area Informatica, una parte
  fondamentale della mia attività è stata la costruzione di nuovi
  insegnamenti pensati per coorti di studenti con buone competenze
  matematiche, ma non all'interno di Corsi di Laurea della classe
  Informatica. Questa attività si è ripetuta più volte, sia in seguito
  all'evoluzione dei corsi di studio di area statistica, che
  successivamente all'interno della laurea triennale in Informatica e
  della laurea magistrale in Data Science.

  La mia modalità di insegnamento è centrata sulla metodologia di active
  learning, che richiede una forte e continua interazione fra docente e
  studente, oltre che fra studenti. Ciò implica normalmente la costruzione
  di problemi che gli studenti devono affrontare, da soli o in gruppo.
  Anche in questo caso l'attività progettuale del corso è innovativa, in
  quanto le pratiche preesistenti e il materiale didattico erano invece
  pensate per modalità più tradizionali, dove gli studenti avevano
  principalmente un ruolo di uditori e la componente di problem solving
  era confinata ad un ruolo secondario.

\subsection{Corsi di dottorato di  ricerca}


\cvitem{2016, 2018, 2020}{Advanced Algorithms, Dottorato di Ricerca in Informatica, Univ.
Milano -- Bicocca (20 ore).
  In questo caso ho dovuto progettare
  completamente l'insegnamento che non era mai stato erogato in
  precedenza.
}

\subsection{Titolarità di Insegnamenti in Corsi di
Studio}


\cvitem{2017-oggi}{%
``Foundations of Computer Science'', Laurea Magistrale in Data Science,
Univ. Milano -- Bicocca. (6 CFU). In questo caso ho dovuto progettare
completamente l'insegnamento, in quanto il corso di studio era di nuova
attivazione.
}
\cvitem{2014-oggi}{%
``Elementi di Bioinformatica'', Laurea Triennale in Informatica, Univ.
Milano -- Bicocca (8 CFU). In questo caso ho dovuto progettare
completamente l'insegnamento che non era mai stato erogato in
precedenza.}
\cvitem{2009--2020}{%
``Basi di Dati'', Laurea Triennale in Statistica e Gestione delle
Informazioni, Laurea Triennale in Scienze Statistiche ed Economiche,
Univ. Milano -- Bicocca (6 CFU).}
\cvitem{2007-oggi}{%
``Bioinformatica'', Laurea Magistrale in Biostatistica, Univ.
Milano -- Bicocca (6 CFU). In questo caso ho dovuto progettare
completamente l'insegnamento che non era mai stato erogato in
precedenza.}
\cvitem{2010--2013}{%
``Algoritmi su stringhe'', Laurea Triennale in Informatica, Univ.
Milano -- Bicocca. In questo caso ho dovuto progettare completamente
l'insegnamento che non era mai stato erogato in precedenza.}
\cvitem{2008}{%
``Strumenti informatici per la statistica M'', Laurea Magistrale in
Biostatistica, blended e-learning, Univ. Milano -- Bicocca. In questo caso
ho dovuto progettare completamente l'insegnamento che non era mai stato
erogato in precedenza.}
\cvitem{2007--2009}{%
``Informatica Applicata S'', Laurea Specialistica in Biostatistica (2
CFU), Univ. Milano -- Bicocca. In questo caso ho dovuto progettare
completamente l'insegnamento che non era mai stato erogato in
precedenza.}
\cvitem{2001--2008}{%
``Laboratorio Statistico-Informatico'', tutte le Lauree triennali della
Facoltà di Scienze Statistiche (6 CFU), Univ. Milano -- Bicocca. In questo
caso ho dovuto riprogettare completamente l'insegnamento che era stato
erogato in precedenza una sola volta, da un docente esterno non
afferente ad alcuna Università.}
\cvitem{2001--2008}{%
``Programmazione e Basi Dati'', tutte le Lauree triennali della Facoltà
di Scienze Statistiche (6 CFU), Univ. Milano -- Bicocca. In questo caso ho
dovuto riprogettare completamente l'insegnamento che era mai stato
erogato in precedenza. I contenuti sono stati nell'ambito di Basi di
Dati (il corso ha poi cambiato nome in Basi di Dati) e in Italia è stato
il primo insegnamento di Basi di Dati dedicato a studenti nell'area
Statistica.}
\cvitem{2008}{%
``Fondamenti di Informatica'', Laurea Magistrale in Biostatistica, Univ.
Milano -- Bicocca. In questo caso ho dovuto progettare completamente
l'insegnamento che non era mai stato erogato in precedenza.}
\cvitem{2006}{%
``Laboratorio di Informatica'', tutte le Lauree triennali della Facoltà
di Scienze Statistiche (6 CFU), Univ. Milano -- Bicocca.}

\subsection{Esercitatore di insegnamenti}

\cvitem{2004--2005}{%
``Bioinformatica: tecniche di base (laboratorio)'', Laurea Magistrale in
Bioinformatica (2 CFU), Univ. Milano -- Bicocca.}

\subsection{Insegnamenti di Master Universitari}

\cvitem{2003}{%
``Fondamenti di Informatica e Elementi di Programmazione'' (Fundamentals
of Computer Science and Elements of Programming), Master di primo
livello in Bioinformatica, Univ. Milano -- Bicocca.}
\cvitem{2001--2002}{%
``Sistemi Informatici e Elementi di Programmazione'', Master di primo
livello in Bioinformatica, Univ. Milano -- Bicocca.}
\cvitem{2012, 2014}{%
``Data Base e Sistemi Informativi, Master di primo livello in
Amministratore di Sistema per la Diagnostica per Immagini, Univ.
Milano -- Bicocca.}
\cvitem{2007, 2010, 2011}{%
``Fondamenti di Informatica'', Master di primo livello in Amministratore
di Sistema per la Diagnostica per Immagini, Univ. Milano -- Bicocca.}

\section{Supervisione}

\subsection{Assegnisti di ricerca}

Sono stato \textbf{responsabile scientifico} dei seguenti assegni di
ricerca:


\cvitem{2018--2019}{%
Murray Patterson. Assegno di ricerca biennale con argomento ``Haplotype
assembly from sequencing reads''. Attualmente Assistant Professor
(Tenure-track) presso Georgia State University.
}\cvitem{2015}{%
Hassan Mahmoud Mohamed Ramadan Mohamed. Assegno di ricerca annuale con
argomento ``Methodology for treatment and data analysis of NGS data for
the detection of alternate splicing events''
}

Inoltre ho collaborato alla formazione dei seguenti assegnisti di
ricerca:

\cvlistitem{Marco Previtali}
\cvlistitem{Stefano Beretta}

\subsection{Dottorandi}

Sono stato \textbf{supervisor} delle seguenti tesi di dottorato in
Informatica

\cvitem{2001--2004}{%
  Riccardo Dondi, ``Computational Problems in the Study of Genomic
  Variations'', 2004 (attualmente Prof.~Associato presso l'Università di
  Bergamo)}
\cvitem{2008--2012}{%
  Stefano Beretta, ``Algorithms for Next Generation Sequencing Data
  Analysis'', 2012 (attualmente Bionformatico presso San Raffaele
  Telethon Institute for Gene Therapy)}
\cvitem{2013--2017}{%
  Marco Previtali, ``Self-indexing for de novo assembly''}
\cvitem{2017--2021}{%
  Simone Ciccolella, ``Algorithms for cancer phylogeny inference'' (attualmente in corso)}

Inoltre ho collaborato attivamente alla formazione dei seguenti
dottorandi:

\cvlistitem{%
  Anna Paola Carrieri, attualmente Research Staff Member presso IBM
  Research Lab, UK}
\cvlistitem{%
  Simone Zaccaria, attualmente Group Leader presso Department of
  Oncology, Univ. College London}
\cvlistitem{%
  Giulia Bernardini, attualmente postdoc press CWI, Amsterdam.}

\subsection{Studenti di laurea  magistrale}

  Sono stato relatore o correlatore di oltre 10 studenti di laurea
  magistrale in Informatica, in Scienze Statistiche ed Economiche, in
  Biostatistica, in Data Science.

  In particolare ho seguito le attività nell'ambito del programma
  \textbf{Exchange mobility EXTRA UE} dei seguenti studenti:

  \cvlistitem{%
  Ramesh Rajaby che ha trascorso 6 mesi presso il gruppo di ricerca del
    prof. Jesper Jansson (Kyoto University, Giappone). Ramesh Rajaby è
    attualmente postdoc alla National University of Singapore.}
    \cvlistitem{%
    Simone Ciccolella che ha trascorso un periodo di 3 mesi presso il
    gruppo di ricerca del prof. Iman Hajirasouliha (Weill Cornell Medical
    College, New York). Simone Ciccolella è attualmente dottorando sotto
    la mia supervisione.}

  \subsection{Studenti di laurea triennale}

  Ho supervisionato le attività di stage e di prova finale di oltre 40
  studenti di Laurea Triennale in Statistica e Gestione delle
  Informazioni, in Informatica, in Scienze Statistiche ed Economiche, in
  Statistica.

  \section{Attività di servizio}

  \subsection{Servizi per l'Ateneo}

\cvitem{2020-oggi}{%
  \textbf{Rappresentante dell'Università degli Studi di Milano -- Bicocca}
  nell'Assemblea Generale della Joint Research Unit ELIXIR IIB, nodo
  italiano di Elixir Europe, l'organizzazione intergovernativa per la
  Bioinformatica.
  }\cvitem{2019-oggi}{%
  \textbf{Vice coordinatore del Dottorato} di Ricerca in Informatica,
  Univ. Milano -- Bicocca
  }\cvitem{2018-oggi}{%
  \textbf{Assicuratore di Qualità} del Corso di Laurea Magistrale in Data
  Science.
  }\cvitem{2019-oggi}{%
  \textbf{Presidente della Commissione Didattica} del Corso di Laurea
  Magistrale in Data Science. (check data)
  }\cvitem{2018-oggi}{%
  \textbf{Responsabile} per l'Università degli Studi di Milano -- Bicocca
  della Convenzione quadro con l'Istituto Nazionale di Genetica
  Molecolare.
  }\cvitem{2020-oggi}{%
  \textbf{Membro del Comitato Scientifico} di Bicocca Ambiente Società
  Economia, in seguito a nomina rettorale.
  }\cvitem{2020-oggi}{%
  \textbf{Membro del Comitato Scientifico} del Master di secondo livello
  ``qOmics: quantitative methods for Omics Data'', Univ.~di Milano -- Bicocca
  e Univ.~di Pavia.
  }\cvitem{2018-oggi}{%
  membro della \textbf{Commissione Paritetica Docenti-Studenti} del
  Dipartimento di Informatica, Sistemistica e Comunicazione
  }\cvitem{2013-oggi}{%
  Membro del collegio docenti del Dottorato di Ricerca in Informatica,
  Univ. Milano -- Bicocca
  }\cvitem{2011-oggi}{%
  \textbf{Membro del Comitato direttivo} del Centro di Produzione
  Multimediale di Ateneo, su nomina del Senato Accademico.
  }\cvitem{2015--2020}{%
  Coordinatore Tecnico Locale per l'Univ. Milano -- Bicocca delle attività
  inerenti Elixir IIB
  }\cvitem{2016-oggi}{%
  \textbf{Responsabile delle attività di training} per giovani ricercatori
  nell'ambito delle iniziative di Elixir IIB. In particolare ho
  organizzato corsi su ``Genome Assembly and Annotation'', ``Data
  Carpentry Workshop'', ``Software Carpentry Workshop'', ``Docker
  Advanced'', ``Exome analysis with Galaxy''.
  }\cvitem{2016--2018}{%
  Membro della commissione orientamento del Dipartimento di Informatica,
  Sistemistica e Comunicazione
  }\cvitem{2002--2012}{%
  \textbf{Referente dell'area Informatica} all'interno della Facoltà di
  Scienze Statistiche. L'incarico è terminato con lo scioglimento della
  Facoltà.
  }\cvitem{2004--2012}{%
  \textbf{Rappresentante della Facoltà} di Scienze Statistiche all'interno
  del comitato di Ateneo per l'Informatica.
  }\cvitem{2010--2012}{%
  \textbf{Delegato del Preside} della Facoltà di Scienze Statistiche per
  l'e-learning.
  }\cvitem{2007--2012}{%
  \textbf{Responsabile} di tutti i laboratori informatici della Facoltà di
  Scienze Statistiche
  }\cvitem{2010--2012}{%
  Membro della commissione per l'elearning della Facoltà di Scienze
  Statistiche.
  }\cvitem{2002--2004}{%
  \textbf{Referente dell'area Informatica} nella commissione della Facoltà
  di Scienze Statistiche che ha disegnato il corso di Laurea Triennale in
  Statistica e Gestione delle Informazioni. Il CdL è ancora attivo.
  }\cvitem{2004}{%
  \textbf{Responsabile} per la Facoltà di Scienze Statistiche del corso
  ``Information Technology For Problem Solving (IT4PS)'', organizzato
  dalla Fondazione CRUI.
  }\cvitem{2003}{%
  \textbf{Responsabile} del corso 167388 ``Laboratorio Complementare di
  Informatica per Statistici'', all'interno del progetto FSE 156165
  ``Progetto Quadro Università degli Studi di Milano -- Bicocca''. Il corso
  ha introdotto competenze informatiche di livello intermedio a studenti
  della Facoltà di Scienze Statistiche. L'iniziativa è stata interamente
  finanziata dall'Unione Europea.}

  \subsection{Servizi per la comunità scientifica}

\cvitem{2013-oggi}{%
  Membro del \textbf{Program Commitee} delle seguenti conferenze
  scientifiche: Computability in Europe (CiE2013, CiE2019, CiE 2020),
  Workshop on Algorithms in Bioinformatics (WABI 2020), ISCB European
  Conference on Computational Biology (ECCB 2019), Combinatorial Pattern
  Matching (CPM 2019), Symposium on String Processing and Information
  Retrieval (SPIRE 2017), Bioinformatics Open Source Conference (BOSC
  2016--2019).
  }\cvitem{2018-oggi}{%
  Executive Officer dello \textbf{Steering Committee} della conferenza
  Computability in Europe
  }\cvitem{1997-oggi}{%
  Revisore di articoli per le seguenti riviste scientifiche: ACM/IEEE
  Transactions on Computational Biology and Bioinformatics, Algorithmica,
  Algorithms, Bioinformatics, Briefings in Bioinformatics, Graphs and
  Combinatorics, Information Processing Letters, INFORMS J. Computing,
  Journal of Computer Science and Technology, Theoretical Computer
  Science, Theory of Computing Systems.
  }\cvitem{2020}{%
  \textbf{Chair} del Workshop Data Structure in Bioinformatics (DSB 2020)
  }\cvitem{2020}{%
  \textbf{Editor} degli atti del convegno Computability in Europe 2020,
  LNCS 12098.
  }\cvitem{2020}{%
  \textbf{Organizzazione}, insieme con Iman Hajirasouliha (Weill-Cornell
  Medical College) della sessione speciale ``Large Scale Bioinformatics
  and Computational Sciences'' della conferenza Computability in Europe
  2020
  }\cvitem{2016--2019}{%
  Membro dell'\textbf{Editorial Board} della rivista scientifica
  ``Advances in Bioinformatics''.
  }\cvitem{2018}{%
  Membro del comitato di valutazione della Tesi di Dottorato di Mattia
  Gastaldello, Univ. Roma la Sapienza.
  }\cvitem{2018}{%
  Membro del comitato di valutazione della Tesi di Dottorato di Luca
  Ferrari, Univ. Milano.
  }\cvitem{2017}{%
  Membro della \textbf{commissione giudicatrice} in valutazione
  comparativa per una posizione di RTDb, Univ. Milano
  }\cvitem{2004}{%
  \textbf{editor di special issue} del Journal of Computer Science and
  Technology.}

  \section{Relazioni a convegni}

  Ho presentato i risultati della mia attività di ricerca a diverse
  conferenze tra cui, in qualità di \textbf{invited speaker} della
  sessione speciale ``Algorithmics for Biology'', alla conferenza
  Computability in Europe 2017, e, in qualità di primo autore (e unico
  autore del mio Ateneo), alla conferenza internazionale Intelligent
  Systems for Molecular Biology (ISMB) 2001: si tratta della principale
  conferenza in Bioinformatica (\textbf{CORE: A, Microsoft Academic: A+}).

  \section{Terza Missione}

  \subsection{Incarichi da enti esterni}


\cvitem{2003--2004}{%
  Incarico per studi relativi agli aspetti informatici relativi al
  progetto INTERREG IIIB (2000--2006) W.E.S.T. WOMEN EAST SMUGGLING
  TRAFFICKING (WP.2.2). \textbf{Committente: Fondazione Ismu-Iniziative e
  Studi sulla Multietnicità}. Sono stato l'unico ricercatore informatico
  coinvolto nel progetto.
  }\cvitem{2005}{%
  Incarico per studi relativi agli aspetti informatici relativi al
  progetto ``Indagine Finalizzata all'Analisi degli Effetti Prodotti dai
  Processi di Regolarizzazione dei Lavoratori Extracomunitari, con
  Particolare Riferimento al Mercato del Lavoro e all'Integrazione Sociale
  nelle Regioni Ob. 1'', finanziato nell'ambito della Misura I.2 FESR
  ``Adeguamento del Sistema di Controllo Tecnologico del Territorio'' del
  PON Sicurezza 2000/2006. \textbf{Committente: Fondazione Ismu-Iniziative
  e Studi sulla Multietnicità}. Sono stato l'unico ricercatore informatico
  coinvolto nel progetto.}

% Publications from a BibTeX file without multibib
%  for numerical labels: \renewcommand{\bibliographyitemlabel}{\@biblabel{\arabic{enumiv}}}% CONSIDER MERGING WITH PREAMBLE PART
%  to redefine the heading string ("Publications"): \renewcommand{\refname}{Articles}
%\nocite{*}
%\bibliographystyle{plain}
%\bibliography{publications}                       % 'publications' is the name of a BibTeX file

% Publications from a BibTeX file using the multibib package
\section{Publications}
\nocitearticle{*}
\bibliographystylearticle{plain}
\bibliographyarticle{article}

\nociteconference{*}
\bibliographystyleconference{plain}
\bibliographyconference{conference}

\nociteinbook{*}
\bibliographystyleinbook{plain}
\bibliographyinbook{inbook}


Milano, \today\\

Gianluca Della Vedova

\end{document}
